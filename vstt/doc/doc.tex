\documentclass[10pt,a4paper]{article}
\usepackage[latin1]{inputenc}
\usepackage[T1]{fontenc}
\usepackage{times}
\usepackage{framed, xcolor}
\usepackage{hyperref}

\fontfamily{lsx}
\fontencoding{T1}

\setlength{\parindent}{0pt}
\setlength{\parskip}{5pt plus 2pt minus 1pt}

\addtolength{\hoffset}{-0.5cm} 
\addtolength{\textwidth}{1cm}
\advance \voffset by -2cm \advance \textheight by 4cm

\pagestyle{headings}

\makeatletter
\def\maketitle{%
  \null
  \thispagestyle{empty}%
  \vfill
  \begin{center}\leavevmode
    \normalfont
    {\LARGE\raggedleft \@author\par}%
    \hrulefill\par
    {\huge\raggedright \textbf{\@title}\par}%
    \vskip 1cm
%    {\Large \@date\par}%
  \end{center}%
  \vfill
  \vfill
  \null
  \cleardoublepage
}
\makeatother

\author{Steffen Wendzel\footnote{www.wendzel.de}}
\title{vstt\footnote{Very Strange Tunneling Tool} documentation}
\date{sep 2006 (with fixes/improvements in Jun 2016, Jun 2020, and Jan-2025)}

\begin{document}

\maketitle
\tableofcontents
%\hrulefill\par
\newpage

\section{Disclaimer}

This tool is for legal educational purposes only! Please also read the LICENSE
file for license details.


\section{Introduction}

Network covert channels enable the stealthy transfer of information over a network. For an introduction, see my free online class on Github: \url{https://github.com/cdpxe/Network-Covert-Channels-A-University-level-Course/}.

\texttt{vstt} is a tunneling tool (primary for TCP connections). It can send
your data via different protocols. Please send your patches if you
port it to new systems or if you fixed a bug.

\textcolor{red}{I wrote this tool in 2006 as a 2nd semester undergraduate student. It might be far from perfect.}

Currently tested systems are:

\begin{itemize}
	\item Linux 2.6.x and newer (i386 or amd64)
	\item OpenBSD 3.x to 4.0-current (i386 and amd64)
	\item SHOULD work too: MacOS, FreeBSD, NetBSD and Solaris (Solaris needs a Makefile modification)
\end{itemize}

\texttt{vstt} can tunnel your data within the following protocols:

\begin{itemize}
	\item NONE (a pseudo protocol) - 99\% done
	\item ICMP - 95\% done
	\item POP3 - 90\% done
	\item DNS - 5\% done (only stub)
\end{itemize}


\section{How to use it?}

\texttt{vstt} receives input from a source, transfers it over a tunnel to another system
running \texttt{vstt}, and outputs the received input to a destination.

\texttt{vstt} accepts input either from a local FIFO or from a TCP stream socket that
you can bind to a port. Similarily, \texttt{vstt} outputs data to a FIFO on the receiver-side or to a TCP
stream socket that you bind to a port.

If you use local FIFOs for input/output, \texttt{vstt} uses the following files:

\begin{verbatim}
binary name | input fifo           | output fifo
----------------------------------------------------------
vstt        | /tmp/.vstt_send2peer | /tmp/.vstt_recvfpeer
\end{verbatim}

You can send data into the connection by writing data into
the input FIFO and you can read received data from the peer
via reading from the output FIFO.

\begin{verbatim}
Q: But I want to use sockets because my TCP app
   (Telnet or SSH for example) uses TCP and not FIFOs.
A: No problem: you have to use the s2f tool included in the
   code -- it bindes a TCP socket to a FIFO!
\end{verbatim}

\section{Examples}

\textcolor{red}{Note: \texttt{vstt} normaly produces one 'connection refused' error every
second if the other peer is not already available. The error messages will
disappear once the connection is established.}

\subsection{Example 1 (without a TCP connection)}

Let us create a simple ICMP tunnel using \texttt{vstt} on two machines.
We want to send a file trough the tunnel and read it with the
shipped tool \texttt{reader}.

This setup requires different parameters to start \texttt{vstt}:

\begin{verbatim}
	-p icmp      <- set the protocol to pop3
	-r n         <- receive data on port n (ignored with ICMP)
	-t m         <- send data to the peer at port m (ignored with ICMP)
	-a x         <- the IP address of the peer
	-m y         <- own IP address
\end{verbatim}
 
 
Setup: We use two Linux machines with the following IPs and will transfer a simple text file.

~\\
Sender:   192.168.2.102\\
Receiver: 192.168.2.101\\
Protocol: ICMP

On the sender, we run the following command (\texttt{-r} and \texttt{-t} are ignored on both computers (as ICMP makes no use of ports) but must be added):

\texttt{sudo ./vstt -p icmp -r 9999 -t 10000 -a 192.168.2.101 -m 192.168.2.102}

(This means to use ICMP; the sender's address is *102, the peer's address is *101)

On the receiver, we run:

\texttt{sudo ./vstt2 -p icmp -r 10001 -t 10002 -a 192.168.2.102 -m 192.168.2.101}

... and in another terminal on the recevier, we start \texttt{reader} that reads the received data from the FIFO:

\texttt{sudo ./reader /tmp/.vstt\_recvfpeer}

Now, the tunnel setup is complete. The data will be transferred via ICMP
from sender to receiver and it will be read out from \textit{/tmp/.vstt\_recvfpeer}.

Finally, we just need to send the actual data that we want to transfer from the sender to the receiver.

On the sender, we simply send the input from a system configuration file to the pipe:

\begin{verbatim}
\$ sudo -i
# cat /etc/resolv.conf > /tmp/.vstt_send2peer
\end{verbatim}

If we now observe the output of the \texttt{reader} on the receiver, we will see
the content of \textit{/etc/resolv.conf} that was transferred via ICMP.


\subsection{Example 2 (tunneling an SSH connection)}

Let's now use an SSH connection between two hosts over
port 80 (e.g.\ because some firewall does not block HTTP but SSH).
We use the protocol '\texttt{none}' because it is fast. `\texttt{none}'
creates nothing but a plain TCP-based tunnel.

\textcolor{red}{Note: You need \textbf{root} access to bind ports below
1024 under most Unix(-like) systems.}

The setup works as follows: both systems start \texttt{vstt} to establish a tunnel they
can communicate through. On the SSH server, we connect our \texttt{vstt}'s FIFO
with the SSH service on port 22 (can be done by using the
\texttt{s2f} tool).

On the client machine, we also use the tool \texttt{s2f} (but in server
mode so that it accepts the SSH client connection to forward it through the tunnel). \texttt{s2f} communicates with the local \texttt{vstt} endpoint through its FIFO.
Finally, we connect to the \texttt{s2f} port using our local SSH client.

Okay, let's start.

Say that `eygo' (192.168.2.20) is the machine with the SSH-Server and that
`hikoki' (192.168.2.21) is the server with the SSH client.

On the first terminal (xterm or a console terminal or whatever), we
start \texttt{vstt}. We receive data on port 80 and send data to port 80 at
the other \texttt{vstt}-endpoint.

\begin{verbatim}
eygo# ./vstt -p none -r 80 -t 80 -a 192.168.2.20 -m your.ip.goes.here
client: connecting to peer ...
server: waiting for connection...
none(or pop3 and so on)_client: connect(): Connection refused
none(or pop3 and so on)_client: connect(): Connection refused
none(or pop3 and so on)_client: connect(): Connection refused
none(or pop3 and so on)_client: connect(): Connection refused
none(or pop3 and so on)_client: connect(): Connection refused
...
...
\end{verbatim}

On the second terminal we start \texttt{s2f}. It will listen on port 10003. We will
connect to this port with the ssh client if the tunnel works.

\begin{verbatim}
eygo# ./s2f -s -p 10003

IMPORTANT NOTE: If you don't want to start s2f by hand,  you can
                also let vstt do that by using -c <port> [-s]
                parameters! Instead of starting vstt+s2f, you could
                start only vstt in this example:
                
                # vstt -p none -r 80 -t 80 -a 192.168.2.20 -c 10003 -s
\end{verbatim}

Please note that the parameter \texttt{-s} means to run as a server and to use the port
given with \texttt{-p} as the listen port instead as the port to connect to.

On eygo, we start \texttt{vstt} too:

\begin{verbatim}
eygo# ./vstt -p none -r 80 -t 80 -a 192.168.2.21 -m your.ip.goes.here
client: connecting to peer ...
server: waiting for connection...
wrapper_tcpserver: connection established => waiting for data...
==> con establ
client: waiting for data from fifo...
\end{verbatim}

And we connect the \texttt{vstt}-FIFOs to the local SSH-Server running on Port 22
via \texttt{s2f}:

\begin{verbatim}
eygo# ./s2f -p 22
connected.

IMPORTANT NOTE: You could alternativeley only start vstt one time without
                calling s2f:
                # ./vstt -p none -r 80 -t 80 -a 192.168.2.21 -c 22
\end{verbatim}

And now, you can connect with SSH to the localhost port 10003 on the first
machine (hikoki).

\begin{verbatim}
hikoki$ ssh user@127.0.0.1 -p 10003
\end{verbatim}

That's it. Your tunnel should be operational now.

\section{Protocols}

\subsection{none}

The `\texttt{none}' protocol is used for a blank tunnel. For example: You sit behind a
firewall that only lets you use port 80 but you want to connect to your
IRC-server at home. You can use the `\texttt{none}' protocol to redirect a connection
over port 80 and then bypass the firewall and enjoy your IRC session.

\subsection{POP3 (alpha quality)}

This is a little bit more advanced. A `\texttt{pop3}' tunnel is slow but it can hide
your data in POP3's RETR-requests. If you want to hide your data a little bit:
use POP3 (or ICMP).

\subsection{ICMP}

If all TCP+UDP ports are blocked, an ICMP tunnel (`\texttt{icmp}') could work anyway. \texttt{vstt} sends
your data as payload in ICMP echo datagrams. \textcolor{red}{\texttt{vstt} can re-send lost packets,
re-calculates the checksum to detect corrupted packets and can
also send big packets from your applications within many small ICMP packets
that will be re-assembled by the peer.} In other words, \texttt{vstt}'s ICMP tunnel implements a simple version of a covert channel-internal control protocol\footnote{see S.~Wendzel, J.~Keller: Hidden and Under Control, in: Annales of Telecommunications, 2014. \url{https://link.springer.com/article/10.1007/s12243-014-0423-x}.}.


\section{Gateways}

You can use different protocol connections between \texttt{vstt}
hosts. Here is an example network using three different \texttt{vstt}
tunnels:

\begin{verbatim}
Client                                   Destination
  |                                         |
  ^                                         ^
Host1 |-----| Host2 |------| Host3 |----| Host4
     ICMP   ICMP    POP3   POP3    NONE NONE
\end{verbatim}

In this scenario, Host2 and Host3 are \texttt{vstt} gateways. As two binaries on the same system would try to utilize the same FIFO files, you need to build \texttt{vstt2} (run \texttt{make vstt2}), which uses different FIFO files, as a second binary on the gateway and direct the output of \texttt{vstt} into the input of \texttt{vstt2}.


\section{Comments, Feedback}

Please send me feedback, typos, bug reports and requests via GitHub
to enhance \texttt{vstt}.

\end{document}

